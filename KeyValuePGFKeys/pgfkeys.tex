\documentclass{scrartcl}
\usepackage{pgfkeys}
\usepackage{listings}

%\newcommand{\inlatex}[1]{\lstinline[language=Tex]|#1|}
\let\inlatex\lstinline
\lstnewenvironment{latex}[1][]
    {\lstset{language={Tex},#1}}
    {}
\newenvironment{result}
    {\noindent\ignorespaces\textbf{Result:}\newline}
    {\par\noindent\ignorespacesafterend}

\usepackage{csquotes}
\begin{document}

% ------------------------------------------------------------------------------
\section{Key-Value Übergabe mit \inlatex{pgfkeys}}

\subsection{Simple usage of keys}

In the following example a handler is assigned to the key
\inlatex{testkey}. Handlers are indicated by names startging with a \enquote{.}. 
This specific handler (\inlatex{.code}) assignes executable latex code to a key.
The execuation of the code is triggered via the command \inlatex$pfgkeys{<key>=<value>}$ 
the code is run with the parameter \inlatex{value}.

\begin{latex}
% set key
\pgfkeys{/testkey/.code=Test1: value '#1'.}
% lookup key 
\pgfkeys{testkey=testvalue}
\end{latex}

\begin{result}
\pgfkeys{/testkey/.code=Test1: value '#1'.}
\pgfkeys{testkey=testvalue}
\end{result}


% ------------------------------------------------------------------------------
\subsection{Define a key with a default value}

Default values for code can be assigned to keys with the handler \inlatex{.default}.
This is very different to \inlatex{.initial}!
\inlatex{.default} assigns default arguments/values for the code assigned to a key, 
whereas \inlatex{.initial} initializes the data values assigned to a key. 

\begin{latex}
% set key
\pgfkeys{/testkey/.code=Test2: value '#1'.}
\pgfkeys{/testkey/.default=defaultvalue}

% lookup
\pgfkeys{/testkey}
\pgfkeys{/testkey=testvalue}
\end{latex}

\begin{result}
% set key
\pgfkeys{/testkey/.code=Test2: value '#1'.}
\pgfkeys{/testkey/.default=defaultvalue}

% lookup
\pgfkeys{/testkey}
\pgfkeys{/testkey=testvalue}
\end{result}

Of course this mechanism can be embedded inside a command:
% # inside newcommand
\begin{latex}
\pgfkeys{/testkey/.code=Test2: value '#1'.}
\pgfkeys{/testkey/.default=defaultvalue}

\newcommand{\testcommand}[1][]{\pgfkeys{#1}}

\testcommand 
\testcommand[testkey=ho]
\end{latex}

\begin{result}
\pgfkeys{/testkey/.code=Test2: value '#1'.}
\pgfkeys{/testkey/.default=defaultvalue}

\newcommand{\testcommand}[1][]{\pgfkeys{#1}}
\testcommand 
\testcommand[testkey=ho]
\end{result}

Also , as the source code of the above examples indicate,
key value assignments can be made local to an environment
they come out of scope after the environment is left.
Either to all instances when specified in the definition of the environment,
or local to a single instance of the environment when specified inside the instance;

% ------------------------------------------------------------------------------
\subsection{Styles: key that are assigned key lists}

Styles are lists of keys. If a style is assigned to a key, the invocation
of the key triggers the invocation of the keys in the list.

\begin{latex}
\pgfkeys{/a/.code=(a:#1)}
\pgfkeys{/b/.code=(b:#1)}
\pgfkeys{/my style/.style={/a=foo,/b=bar,/a=#1}}

\pgfkeys{/my style=wow}
\end{latex}

\begin{result}
\pgfkeys{/a/.code=(a:#1)}
\pgfkeys{/b/.code=(b:#1)}
\pgfkeys{/my style/.style={/a=foo,/b=bar,/a=#1}}

\pgfkeys{/my style=wow}
\end{result}


% ------------------------------------------------------------------------------
\subsection{Styles: Style and code are mutually exclusive}

By default, style and code handlers for a key are mutually exclusive --
a key is assigned to code or to a style.

\begin{latex}
\pgfkeys{/c/.code=c is executed}
\pgfkeys{/c/a/.code=a is executed}
\pgfkeys{/c/b/.code=b is executed}
\pgfkeys{/c/.style={c/.cd,a,b}}

\pgfkeys{c}
\end{latex}

\begin{result}
\pgfkeys{/c/.code=c is executed}
\pgfkeys{/c/a/.code=a is executed}
\pgfkeys{/c/b/.code=b is executed}
\pgfkeys{/c/.style={c/.cd,a,b}}

\pgfkeys{c}
\end{result}


% ------------------------------------------------------------------------------
\subsection{Family: Shortcut to setting a folder}

The handler \inlatex{.is family} defines a key as a shortcut to send the current
folder (style comes from tikz... folders in tikz define different syling for 
elements).

\begin{latex}
\pgfkeys{/c/a/.code=a}
\pgfkeys{/c/b/.code=b}
\pgfkeys{/c/.is family}

% set current path to /c/, then access a and b by name
\pgfkeys{c, a, b}
\end{latex}

\begin{result}
\pgfkeys{/c/a/.code=a}
\pgfkeys{/c/b/.code=b}
%\pgfkeys{/c/.style={/c/a,/c/b}}
\pgfkeys{/c/.is family}

\pgfkeys{c, a, b}
\end{result}

The handler \inlatex{is family} is just a synonym:
\inlatex{<key>/.is family} is the same as \inlatex{<key>/.style=<key>/.cd}.

Because styles and commands are mutually exclusive by default,
a key can either act as a directory shortcut via \inlatex{is family} or 
execude code via \inlatex{.code}.
(Thus it seems there are ways around this).

% ------------------------------------------------------------------------------
\subsection{Every folder can have associated code and handlers}

With modeling keys like folders, every folder can have associated code,
and still have subkeys:

\begin{latex}
\pgfkeys{/c/.code=(/c:#1)}
\pgfkeys{/c/a/.code=(/c/a:#1)}
\pgfkeys{/c/b/.code=(/c/b:#1)}

\pgfkeys{/c=that}
\pgfkeys{/c/.cd, a=this}
\end{latex}

\begin{result}
\pgfkeys{/c/.code=(/c:#1)}
\pgfkeys{/c/a/.code=(/c/a:#1)}
\pgfkeys{/c/b/.code=(/c/b:#1)}

\pgfkeys{/c=that}
\pgfkeys{/c/.cd, a=this}
\end{result}


% ------------------------------------------------------------------------------
\subsection{Defining Value-, Macro-, If- and Choice Keys}

\subsubsection{Keyhandler `.initial'}

The handler \inlatex{.initial} is used to assign an initial value to a key.
It should not be confused with the handler \inlatex{.default}.
\inlatex{.default} specifies an default argument for code that it assigned 
to a key with \inlatex{.code}. 

The following example shows \inlatex{.initial} together with \inlatex{.get}.

\begin{latex}
\pgfkeys{/a/.initial=one}
\pgfkeys{/b/.initial=two}

\newcommand\test[1][]{
\pgfkeys{#1}
\pgfkeys{/a/.get=\width}
\pgfkeys{/b/.get=\height}

\width \height

\test \test[a=testa] \test[a=testa, b=testb]
}
\end{latex}

\begin{result}
\pgfkeys{/a/.initial=inita}
\pgfkeys{/b/.initial=initb}

\newcommand\test[1][]{
\pgfkeys{#1}
\pgfkeys{/a/.get=\width}
\pgfkeys{/b/.get=\height}

\width \height
}
\test \test[a=testa] \test[a=testa, b=testb]
\end{result}


\subsubsection{keyhandler `.get'}

The handler \inlatex{.get} executes a \inlatex{let} command so that the contents 
of a key are stored in a macro. 

\begin{latex}
\pgfkeys{/my key/.initial=red}
\pgfkeys{/my key=blue}
\pgfkeys{/my key/.get=\mymacro}
\mymacro
\end{latex}

\begin{result}
\pgfkeys{/my key/.initial=red}
\pgfkeys{/my key=blue}
\pgfkeys{/my key/.get=\mymacro}
\mymacro
\end{result}

\subsubsection{keyhandler `.add'}

Adds the \inlatex{<prefix value>} to the beginning and the 
\inlatex{<append value>} at the end of the value stored in key.

\begin{latex}
\pgfkeys{/my key/.initial=red}
\pgfkeys{/my key/.add={a }{ flower}}
\pgfkeys{my key}
\end{latex}

\begin{result}
\pgfkeys{/my key/.initial=red}
\pgfkeys{/my key/.add={a }{ flower}}
\pgfkeys{my key}
\end{result}

% same stuff
% .prefix
% .append

\subsubsection{keyhandler `.link'}

Somewhat like soft links in unix


\subsubsection{keyhandler `.store in'}

key=value should cause value to be stored in some macro
typically you could just as well store the value in the key itself.

The difference to \inlatex{.get} is, that will get saves the value of a key
in a macro, \inlatex{.store in} stores the result of the computation associated 
with the code in a macro... (is that correct?)

\begin{latex}
\pgfkeys{/text/.store in=\mytext} % declare safe macro
\pgfkeys{/text=Hello world!} % saves stuff in macro
\mytext
\end{latex}

\begin{result}
\pgfkeys{/text/.store in=\mytext} % declare safe macro
\pgfkeys{/text=Hello world!} % saves stuff in macro
\mytext
\end{result}

\subsubsection{keyhandler `.is if'}

\subsubsection{keyhandler `.is choice'}

\end{document}

